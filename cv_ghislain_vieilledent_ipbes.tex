\documentclass[10pt,a4paper,sans]{moderncv}
\renewcommand*{\bibliographyhead}[1]{}
\moderncvstyle{classic}      
\moderncvcolor{green}             
\usepackage[utf8]{inputenc}
\usepackage[scale=0.8]{geometry}
\usepackage{eurosym}
\usepackage{calc}
\usepackage{lastpage}

\firstname{Ghislain}
\familyname{Vieilledent}
\title{Ecology -- Applied Statistics}              
\address{Cirad, UMR AMAP}{F-34398 Montpellier, France}
%\mobile{+39.329.457.2273}
\mobile{+33.(0)4.67.61.49.24} 
%\mobile{+261.(0)32.07.235.34}                    
\email{ghislain.vieilledent@cirad.fr}                      
\homepage{ecology.ghislainv.fr}
%\extrainfo{born in 1979 in Nantes (40 yrs old)}
\photo[70pt][0.4pt]{Identite_Australie.png}
%\photo[70pt][0.4pt]{ghislain_5_300x300.png}
\rfoot{\textcolor{gray}{\textit{\small{\thepage/\pageref{LastPage}}}}} 

% to show numerical labels in the bibliography; only useful if you make citations in your resume
\newcounter{enumiv_gv}
\setcounter{enumiv_gv}{0}
\makeatletter
\renewcommand*{\bibliographyitemlabel}{\@biblabel{\arabic{enumiv}}}
\makeatother

\begin{document}
\makecvtitle

\section{Positions}

\cventry{2009 -- pres.}{Research scientist}{Cirad, UMR AMAP}{Montpellier (France)}{Ecology and applied statistics}{}
\cventry{2015 -- 2018}{Seconded National Expert (SNE)}{European Commission, Joint Research Center} {Ispra (Italie)}{Modelling and forecasting anthropogenic deforestation in the tropics}{}
\cventry{2009 -- 2012}{Detached researcher in Madagascar}{Cirad, dP Forêts et Biodiversité} {Antananarivo (Madagascar)}{Functioning and conservation of tropical forest ecosystems in Madagascar}{}

\section{Research topics}

\cvline{\textbf{Main topics}}{\textbf{Community ecology:} competition between species, intraspecific variability, and demography of rare species; \textbf{Conservation biology:} role of tropical forests in the global carbon cycle and species vulnerability to land-use and climate change; \textbf{Statistics applied to ecology:} advanced statistical models for modelling species distribution and tropical anthropogenic deforestation.}
\cvline{\textbf{Land-use change modelling}}{During my stay at the Joint Research Center of the European Commission as a Seconded National Expert (2015--2018), I have developed a methodolody and new statistical and informatic tools to model and forecast deforestation at the pantropical scale.}

\section{Background and diploma}

\cvline{2009}{\textbf{Cemagref de Grenoble, AgroParisTech.} \emph{PhD thesis} in statistics applied to forest ecology. Tree species coexistence mecanisms in mixed mountain forests (Norway Spruce and silver Fir).}

\cvline{2004}{\textbf{Ecole Nationale Supérieure Agronomique de Rennes
  (ENSAR).} \emph{Diploma of Agronomy}. Agronomy, quantitative ecology and conservation biology.}

\section{Funded research projects}

\cvline{2020 -- 2023}{\textbf{METRADICA} project (co-PI), Mechanistic traits to predict shifts in tree species abundance and distribution with climate change in the Amazonian forest, \textbf{Labex CEBA}, 200 000 \euro, \url{http://www.labex-ceba.fr/projets-strategiques}}

\cvline{2020 -- 2022}{\textbf{INTRACO} project (PI), Unravelling the role of intraspecific variability in tree species coexistence in tropical forest, \textbf{FRB-Cesab}, 4 international workshops, \url{https://frama.link/intraco}}

\cvline{2019 -- 2022}{\textbf{RELIQUES} project (in charge of WP1), Effect of forest fragmentation on biodiversity in ultramafic forests of New-Caledonia, \textbf{CNRT}, 251 380 \euro, \url{https://cnrt.nc/reliques}}

\cvline{2014 -- 2019}{\textbf{BioSceneMada} project (coordinator), Biodiversity scenario under the effects of both climate change and deforestation in Madagascar, \textbf{FRB--FFEM}, 116 748 \euro, \url{https://bioscenemada.cirad.fr}}

\cvline{2010 -- 2012}{\textbf{REDD+} projects, Contribution of Madagascar tropical forests to the global carbon cycle, \textbf{AFD, GoodPlanet, WWF}, 69 210 \euro}

\section{Software development}

\cvline{\textbf{Statistical R packages}}{\textbf{hSDM} and \textbf{jSDM} for advanced species distribution modelling. Contributions to \textbf{MCMCpack} for generalized linear mixed models (glmm).}

\cvline{\textbf{Python module}}{\textbf{forestatrisk} for modelling and forecasting deforestation in the tropics: \url{https://github.com/ghislainv/forestatrisk}} 

\section{Scientific publications}

\vspace{0.3cm}
\subsection{Summary}

\cvline{}{Since 2009, 38 scientific articles and 3 book chapters, \textbf{H-index}: 25 (Google Scholar), 19 (Web of Science). Complete list at: \url{https://ecology.ghislainv.fr/publications}}

\vspace{0.3cm}
\subsection{Publications in link with the IPBES call}

\begin{thebibliography}{5}

\setcounter{enumiv}{9}
\bibitem{Strona2018}
\textbf{Strona G., S.~D. Stringer, \underline{G. Vieilledent}, Z. Szantoi, J. Garcia-Ulloa and S.~A. Wich.}
\newblock 2018.
\newblock Small room for compromise between oil palm cultivation and primate conservation in Africa.
\newblock{\em Proceedings of the National Academy of Sciences PNAS}, 115(35):8811--8816.
\newblock \textcolor{gray}{[doi: \href{https://doi.org/10.1073/pnas.1804775115}{10.1073/pnas.1804775115}]}

\setcounter{enumiv}{8}
\bibitem{Dezecache2017}
\textbf{Dezécache C., J.-M. Salles, \underline{G. Vieilledent}, and B. Hérault.}
\newblock 2017.
\newblock Moving forward socio-economically focused models of deforestation.
\newblock{\em Global Change Biology}, 23(9): 3484-3500.
\newblock \textcolor{gray}{[doi: \href{https://doi.org/10.1111/gcb.13611}{10.1111/gcb.13611}]}

\setcounter{enumiv}{7}
\bibitem{Vieilledent2016}
\textbf{\underline{Vieilledent G.}, O. Gardi, C. Grinand, C. Burren, M. Andriamanjato, C. Camara, C. J. Gardner, L. Glass, A. Rasolohery, H. Rakoto Ratsimba, V. Gond, and J.-R. Rakotoarijaona.} 
\newblock 2016.
\newblock Bioclimatic envelope models predict a decrease in tropical forest carbon stocks with climate change in Madagascar.
\newblock{\em Journal of Ecology}, 104:703--715.
\newblock \textcolor{gray}{[doi: \href{https://doi.org/10.1111/1365-2745.12548}{10.1111/1365-2745.12548}]}

\setcounter{enumiv}{6}
\bibitem{Vieilledent2013}
\textbf{\underline{Vieilledent G.}, C. Cornu, A. Cuní Sanchez, J.-M. Leong Pock-Tsy and P. Danthu} 
\newblock 2013.
\newblock Vulnerability of baobab species to climate change and effectiveness of the protected area network in Madagascar: towards new conservation priorities.
\newblock{\em Biological Conservation}, 166:11-22.
\newblock \textcolor{gray}{[doi: \href{https://doi.org/10.1016/j.biocon.2013.06.007}{10.1016/j.biocon.2013.06.007}]}

\setcounter{enumiv}{5}
\bibitem{Vieilledent2013a}
\textbf{\underline{Vieilledent G.}, C. Grinand and R. Vaudry} 
\newblock 2013.
\newblock Forecasting deforestation and carbon emissions in tropical developing countries facing demographic expansion: a case study in Madagascar.
\newblock{\em Ecology and Evolution}, 3:1702-1716.
\newblock \textcolor{gray}{[doi: \href{https://doi.org/10.1002/ece3.550}{10.1002/ece3.550}]}

\end{thebibliography}

\vspace{0.3cm}
\subsection{Other high impact publications}

\begin{thebibliography}{5}

\setcounter{enumiv}{4}
\bibitem{Vancutsem2020}
\textbf{Vancutsem C. and F. Achard and J.-F. Pekel and \underline{G. Vieilledent} and S. Carboni and D. Simonetti and J. Gallego and L. Aragao and R. Nasi.}
\newblock 2020.
\newblock Long-term (1990-2019) monitoring of tropical moist forests dynamics. 
\newblock{\em bioRxiv}, 2020.09.17.295774.
\newblock \textcolor{gray}{[doi: \href{https://doi.org/10.1101/2020.09.17.295774}{10.1101/2020.09.17.295774}]}

\setcounter{enumiv}{3}
\bibitem{Vieilledent2018}
\textbf{\underline{Vieilledent G.}, C. Grinand, F. A. Rakotomalala, R. Ranaivosoa, J.-R. Rakotoarijaona, T. F. Allnutt, and F. Achard.}
\newblock 2018.
\newblock Combining global tree cover loss data with historical national forest-cover maps to look at six decades of deforestation and forest fragmentation in Madagascar. 
\newblock{\em Biological Conservation}, 222: 189-197.
\newblock \textcolor{gray}{[doi: \href{https://doi.org/10.1016/j.biocon.2018.04.008}{10.1016/j.biocon.2018.04.008}]}

\setcounter{enumiv}{2}
\bibitem{Kunstler2016}
\textbf{Kunstler G., D. Falster, D. Coomes, F. Hui, R. Kooyman, D. Laughlin, L. Poorter, M. Vanderwel, \underline{G. Vieilledent}, S.~J. Wright, M. Aiba, C. Baraloto, J. Caspersen, J. H. C. Cornelissen, S. Gourlet-Fleury, M. Hanewinkel, B. Hérault, J. Kattge, H. Kurokawa, Y. Onoda, J. Penuelas, H. Poorter, M. Uriarte, S. Richardson, P. Ruiz-Benito, I.-F. Sun, G. Ståhl, N. Swenson, J. Thompson, B. Westerlund, C. Wirth, M. Zavala , H. Zeng, J. Zimmerman, N. Zimmermann, and M. Westoby.} 
\newblock 2016.
\newblock Plant functional traits have globally consistent effects on competition.
\newblock{\em Nature}, 529:204--207.
\newblock \textcolor{gray}{[doi: \href{https://doi.org/10.1038/nature16476}{10.1038/nature16476}]}

\setcounter{enumiv}{1}
\bibitem{Chave2015}
\textbf{Chave J., M. Réjou-Méchain, A. Búrquez, E. Chidumayo, M. S. Colgan, \ldots, and \underline{G. Vieilledent}.} 
\newblock 2014.
\newblock Improved allometric models to estimate the aboveground biomass of tropical trees. 
\newblock{\em Global Change Biology}, 20:3177--3190.
\newblock \textcolor{gray}{[doi: \href{https://doi.org/10.1111/gcb.12629}{10.1111/gcb.12629}]}

\setcounter{enumiv}{0}
\bibitem{Kunstler2012}
\textbf{Kunstler G., S. Lavergne, B. Courbaud, W. Thuiller,
  \underline{G. Vieilledent}, N.~E. Zimmermann, J. Kattge and D.~A. Coomes.}
\newblock 2012.
\newblock Competitive interactions between forest trees are driven by species'
  trait hierarchy, not phylogenetic or functional similarity: implications for
  forest community assembly.
\newblock {\em Ecology Letters}, 15(8):831--840.
\newblock \textcolor{gray}{[doi: \href{https://doi.org/10.1111/j.1461-0248.2012.01803.x}{10.1111/j.1461-0248.2012.01803.x}]}

\end{thebibliography}

\end{document}
