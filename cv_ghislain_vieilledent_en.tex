\documentclass[10pt,a4paper,sans]{moderncv}
\renewcommand*{\bibliographyhead}[1]{}
\moderncvstyle{classic}      
\moderncvcolor{green}             
\usepackage[utf8]{inputenc}
\usepackage[francais, english]{babel}
\usepackage[scale=0.8]{geometry}
\usepackage{eurosym}
\usepackage{calc}
\usepackage{lastpage}
\usepackage{amsmath}

\firstname{Ghislain}
\familyname{Vieilledent}
\title{Ecology -- Applied Statistics}              
\address{UMR AMAP}{2196, bd. de la Lironde}{34980 Montferriez-sur-Lez (FR)}
\mobile{+33.(0)4.67.61.49.24}                   
\email{ghislain.vieilledent@cirad.fr}                      
\homepage{ecology.ghislainv.fr}
\extrainfo{born in 1979 in Nantes (FR)}
\photo[70pt][0.4pt]{Identite_Australie.png}
%\photo[70pt][0.4pt]{ghislain_5_300x300.png}
\rfoot{\textcolor{gray}{\textit{\small{\thepage/\pageref{LastPage}}}}} 

% to show numerical labels in the bibliography; only useful if you make citations in your resume
\newcounter{enumiv_gv}
\setcounter{enumiv_gv}{0}
\makeatletter
\renewcommand*{\bibliographyitemlabel}{\@biblabel{\arabic{enumiv}}}
\makeatother

% doi
\newcommand\doi[1]{[doi: \href{https://doi.org/#1}{#1}]}

\begin{document}
\makecvtitle

\vspace{-0.7cm}
\textit{\small{CV updated on \today.}}

\section{Positions}

\cventry{2019 -- pres.}{Research scientist}{Cirad, UMR AMAP}{Montpellier (France)}{Study of tropical forests}{}
\cventry{2015 -- 2018}{Seconded National Expert (SNE)}{European Commission, Joint Research Center} {Ispra (Italie)}{Modelling and forecasting anthropogenic deforestation in the tropics}{}
\cventry{2009 -- 2012}{Detached researcher in Madagascar}{Cirad, dP Forêts et Biodiversité} {Antananarivo (Madagascar)}{Functioning and conservation of tropical forest ecosystems in Madagascar}{}
\cventry{2009 -- 2018}{Research scientist}{Cirad, UPR Forêts et Sociétés} {Montpellier (France)}{Study of tropical forests}{}

\section{Research topics}

\cvline{}{\textbf{Climate change and tropical forests}. Anticipating the impact of climate change on tropical forests and assessing the role of tropical forests in the global carbon cycle.}
\cvline{}{\textbf{Community ecology}. Explaining the distribution of tree species and their coexistence through the analysis of tropical forest inventory data and the use of theoretical models.}
\cvline{}{\textbf{Sustainable management of tropical forests}. Improving the realism of forest dynamics models so that they can be used for forest management.}
\cvline{}{\textbf{New statistical methods for ecology}. Developing new statistical approaches and computer tools to answer new questions in ecology.}

\section{Background and diploma}

\cvline{2006 -- 2009}{\textbf{Cemagref de Grenoble (now INRAE), AgroParisTech.} \emph{PhD thesis}. Structuring uncertainty and variability in allometric and demographic functions used in forest dynamics models to identify coexistence mechanisms of tree species in the mountain forests of the Alps.}

\cvline{2004 -- 2006}{\textbf{Chambre d'Agriculture de la Lozère, CLAPE-LR, FREDEC-MP.} Engineer in agronomy in charge of agronomic and environmental studies.}

\cvline{2003 -- 2004}{\textbf{National Museum of Natural History (MNHN), INA-PG, Paris VII.} \emph{Master.} ``Environment: natural areas, technics, and societies''. Conservation and sustainable development.}

\cvline{2001}{\textbf{Escuela Técnica Superior de Ingenieros Agronomos y Montes (ETSIAM), Córdoba (Spain).} Forestry and conservation of Mediterranean habitats.}

\cvline{2000 -- 2003}{\textbf{Ecole Nationale Supérieure Agronomique de Rennes (ENSAR).} \emph{Diploma of Engineer in Agronomy.}  Agronomy, quantitative ecology, and biodiversity conservation.}

\cvline{1996 -- 1999}{\textbf{Lycée Georges Clémenceau, Nantes.} 1997 -- 1999: Student in preparatory classes BCPST (biology and mathematics). 1996 -- 1997: Baccalauréat S with a major in mathematics.}

\section{Programming and foreign languages}

\cvline{\textbf{Programming}}{R, Python, C/C++, Bash, SQL}

\cvline{\textbf{GIS}}{GRASS GIS, QGis}

\cvline{\textbf{OS}}{Linux (Debian)}

\cvline{\textbf{English}}{Proficient (TOEFL, level C). Two-month stay in North Carolina (Duke University).}

\cvline{\textbf{Spanish}}{Intermediate (level B). Six-month stay both in Spain and Peru.}\enlargethispage{\baselineskip}

\cvline{\textbf{Italian}}{Intermediate (level B). Detached researcher in Italy for three years.}

\newpage

\section{Research projects}

\cvline{}{Since 2009: coordination of 3 scientific projects, in charge of work-packages for 1 project, participation in 4 other projects, and 6 consultancy works.}
\cvline{\homepagesymbol}{Complete description: \url{https://ecology.ghislainv.fr/research.html\#projects}.}

\vspace{0.3cm}
\subsection{Project coordination}

\cvline{2020 -- 2023}{\textbf{METRADICA}, Mechanistic traits to predict shifts in tree species abundance and distribution with climate change in the Amazonian forest, \textbf{Labex CEBA}, 200,000 \euro. \url{http://www.labex-ceba.fr/projets-strategiques}}

\cvline{2020 -- 2022}{\textbf{INTRACO}, Role of intraspecific variability in tree species coexistence in tropical forests, \textbf{FRB-Cesab -- sDiv}, 4 international workshops ($\sim$ 40,000 \euro). Webpage on \href{https://www.fondationbiodiversite.fr/en/the-frb-in-action/programs-and-projects/le-cesab/intraco/}{\texttt{Cesab}'s website.}}

\cvline{2014 -- 2019}{\textbf{BioSceneMada}, Biodiversity scenarios under the effects climate change and deforestation in Madagascar, \textbf{FRB -- FFEM}, 116,748 \euro. \url{https://bioscenemada.cirad.fr}}

\vspace{0.3cm}
\subsection{In charge of work-packages}

\cvline{2019 -- 2022}{\textbf{RELIQUES}, Effect of forest fragmentation on biodiversity in ultramafic forests of New-Caledonia, \textbf{CNRT}, 251,380 \euro. Coordination: Philippe Birnbaum (Cirad, UMR AMAP). \url{https://cnrt.nc/reliques}}

\vspace{0.3cm}
\subsection{Participation in projects}

\cvline{2023 -- 2026}{\textbf{EDENE} project, "Ecological Dynamics behind species Extinctions in Novel Ecosystems", \textbf{ANR}, 302,165 \euro. Coordination: Robin Pouteau (IRD, UMR AMAP).}

\cvline{2022 -- 2025}{\textbf{ALT} project, "Amazonian Landscapes in Transition", \textbf{ANR}, 659,799 \euro. Coordination: Jérôme Chave (CNRS, University of Toulouse).}

\cvline{2022 -- 2025}{\textbf{GUARDEN} project, "safeGUARDing biodivErsity aNd critical ecosystem services across sectors and scales", \textbf{European Union}, Horizon Europe, 4.5M \euro. Coordination: Pierre Bonnet (Cirad, UMR AMAP).}

\cvline{2019 -- 2023}{\textbf{GAMBAS} project, "Generating Advances in Modeling Biodiversity And ecosystem Services (GAMBAS): statistical improvements and ecological relevance of joint species distribution models", \textbf{ANR}, 569,033 \euro. Coordination: Frédéric Mortier (Cirad, UPR Forêts et Sociétés).}

\vspace{0.3cm}
\subsection{Consultancy works}

\cvline{2023}{Validation of global forest cover change maps in New Caledonia for the periods 2000--2010--2020 using photo-interpretation, \textbf{Oeil}, 12,000 \euro.}

\cvline{2022}{Developing the \textbf{riskmapjnr} Python package for obtaining maps of deforestation risk following the JNR (Jurisdictional Nested REDD+) methodology, \textbf{FAO}, 28,000 \euro.}

\cvline{2012 -- 2013}{Technical assistance to Office National pour l'Environnement for implementing REDD+ in Madagascar, \textbf{AFD}, 290,210 \euro.}

\cvline{2012}{Forest inventories and development of height-diameter allometric equations for Madagascar dry forests, \textbf{WWF}, 3500 \euro.}

\cvline{2011 -- 2012}{Modelling deforestation (intensity and location of the deforestation) in Madagascar, \textbf{GoodPlanet}, 9000 \euro.}

\cvline{2010 -- 2011}{Developing biomass allometric equations for Madagascar forests, \textbf{WWF -- GoodPlanet}, 8750 \euro.}

\newpage

\section{Software development}

\cvline{}{Since 2009: Development and maintenance of 4 R packages and 3 Python packages. Contribution to 1 R package. Provision of 70 public \href{https://github.com/ghislainv}{\texttt{GitHub}} repositories and 6 \href{https://dataverse.cirad.fr/dataverse/CIRAD/?q=\%28authorName\%3Aghislain+authorName\%3Avieilledent\%29}{\texttt{Cirad Dataverse}} repositories.}
\cvline{\homepagesymbol}{Complete description: \url{https://ecology.ghislainv.fr/software.html}.}

\vspace{0.3cm}
\subsection{R packages}

\cvline{\textbf{gecevar}}{The \textbf{gecevar} package extracts climatic or other environmental variables (topography, soils) from various online datasets for any region of interest and resolution ($\geq$ 250m) specified by the user. \url{https://ecology.ghislainv.fr/gecevar}}

\cvline{\textbf{jSDM}}{The \textbf{jSDM} package provides functions for estimating parameters of joint species distribution models. \url{https://ecology.ghislainv.fr/jSDM}}

\cvline{\textbf{hSDM}}{The \textbf{hSDM} package allows estimating parameters of hierarchical Bayesian species distribution models. \url{https://ecology.ghislainv.fr/hSDM}}

\cvline{\textbf{twoe}}{\textbf{twoe} (2e) is the name of an R package and a Capsis module for estimating demographic parameters of tree species and simulating forest dynamics from permanent forest inventory plots. \url{https://twoe.sourceforge.net}}

\cvline{\textbf{MCMCpack}}{(contribution) Functions developed within the \textbf{MCMCpack} package allow estimating parameters of generalized linear mixed effect models (glmm). \url{http://cran.r-project.org/package=MCMCpack}}

\vspace{0.3cm}
\subsection{Python packages}

\cvline{\textbf{forestatrisk}}{The \textbf{forestatrisk} package provides functions to model deforestation and predict forest cover changes under various scenarios of deforestation in tropical countries. \url{https://ecology.ghislainv.fr/forestatrisk}}

\cvline{\textbf{pywdpa}}{The \textbf{pywdpa} package extracts informations from the World Database on Protected Areas (WDPA) for a given country. \url{https://ecology.ghislainv.fr/pywdpa}}

\cvline{\textbf{riskmapjnr}}{The \textbf{riskmapjnr} package provides functions to derive map of the deforestation risk following the JNR (Jurisdictional Nested REDD+) methodology. \url{https://ecology.ghislainv.fr/riskmapjnr}}

\section{Teaching and supervision}

\cvline{}{Since 2009: 2 post-docs ($+$ 2 in collaboration), 4 PhD students ($+$3 in collaboration), 14 Master students.}
\cvline{\homepagesymbol}{Complete description: \url{https://ecology.ghislainv.fr/people.html}.}

\cvline{2019}{Using species distribution modelling for predicting their vulnerability to climate change. Capacity building workshop. Students, technicians, and engineers. Campus Numérique Francophone. Antananarivo (Madagascar). 2d.}

\cvline{2018--2019}{Modelling and forecasting deforestation in the tropics. Capacity building workshop. Students, technicians, and engineers. CeRSAE FOFIFA and Campus Numérique Francophone. Antananarivo (Madagascar). 5d.}

\cvline{2012}{Statistical regressions and biomass allometric models. Learning how to use the R software. Capacity building workshop. Students, technicians, and engineers. Office National pour l'Environnement. Antananarivo (Madagascar). 3.5d.}

\cvline{$<$ 2009}{Theoretical models in ecology. Theoretical classes. Master students in Biology, Ecology, and Environment. University Joseph Fourier. Grenoble (France). 3h.}

\cvline{$<$ 2009}{Theoretical models in ecology. Practical work. Solving systems of differential equations for studying species coexistence mechanisms. Master students in Biology, Ecology, and Environment. University Joseph Fourier. Grenoble (France). 16h.}

\cvline{$<$ 2009}{Forest dynamics models. Practical work. Learning how to use the Capsis software with the Samsara module. Master students. FIF-ENGREF. Nancy (France). 4h.}

\newpage

\section{Scientific publications}

\cvline{}{Since 2009: 47 articles published in scientific journals, 4 pre-prints, 4 book chapters, and 3 popular science articles. H-index: 32 (GScholar), 24 (WoS).}
\cvline{\homepagesymbol}{Complete list at: \url{https://ecology.ghislainv.fr/publications.html}.}

\vspace{0.3cm}
\subsection{Selection of five publications as first author}

\begin{thebibliography}{5}

\setcounter{enumiv}{4}
\bibitem{Vieilledent2022}
\textbf{\underline{Vieilledent G.}, C. Vancutsem, C. Bourgoin, P. Ploton, P. Verley, and F. Achard.} 
\newblock 2022.
\newblock Spatial scenario of tropical deforestation and carbon emissions for the 21$^{\text{st}}$ century.
\newblock{\em bioRxiv}, 485306.
\newblock \textcolor{gray}{\doi{10.1101/2022.03.22.485306}}

\setcounter{enumiv}{3}
\bibitem{Vieilledent2018}
\textbf{\underline{Vieilledent G.}, C. Grinand, F. A. Rakotomalala, R. Ranaivosoa, J.-R. Rakotoarijaona, T. F. Allnutt, and F. Achard.} 
\newblock 2018.
\newblock Combining global tree cover loss data with historical national forest-cover maps to look at six decades of deforestation and forest fragmentation in Madagascar.
\newblock{\em Biological Conservation}, 222: 189--197.
\newblock \textcolor{gray}{\doi{10.1016/j.biocon.2018.04.008}}

\setcounter{enumiv}{2}
\bibitem{Vieilledent2016}
\textbf{\underline{Vieilledent G.}, O. Gardi, C. Grinand, C. Burren, M. Andriamanjato, C. Camara, C. J. Gardner, L. Glass, A. Rasolohery, H. Rakoto Ratsimba, V. Gond, and J.-R. Rakotoarijaona.} 
\newblock 2016.
\newblock Bioclimatic envelope models predict a decrease in tropical forest carbon stocks with climate change in Madagascar.
\newblock{\em Journal of Ecology}, 104: 703--715.
\newblock \textcolor{gray}{\doi{10.1111/1365-2745.12548}}

\setcounter{enumiv}{1}
\bibitem{Vieilledent2012}
\textbf{\underline{Vieilledent G.}, R. Vaudry, S.~F.~D. Andriamanohisoa, O.~S. Rakotonarivo, H.~Z.
  Randrianasolo, H.~N. Razafindrabe, C. Bidaud Rakotoarivony, J. Ebeling, and M. Rasamoelina.}
\newblock 2012.
\newblock A universal approach to estimate biomass and carbon stock in tropical
  forests using generic allometric models.
\newblock {\em Ecological Applications}, 22(2): 572--583.
\newblock \textcolor{gray}{\doi{10.1890/11-0039.1}}

\setcounter{enumiv}{0}
\bibitem{Vieilledent2010b}
\textbf{\underline{Vieilledent G.}, B. Courbaud, G. Kunstler, J.-F. Dhôte, and J.~S. Clark.}
\newblock 2010.
\newblock Individual variability in tree allometry determines light resource
  allocation in forest ecosystems: a hierarchical Bayesian approach.
\newblock {\em Oecologia}, 163(3): 759--773.
\newblock \textcolor{gray}{\doi{10.1007/s00442-010-1581-9}}

\end{thebibliography}

\vspace{0.3cm}
\subsection{Selection of five publications as coauthor}

\begin{thebibliography}{5}

\setcounter{enumiv}{4}
\bibitem{Girard-Tercieux2023}
\textbf{Girard-Tercieux C., I. Maréchaux, A. T. Clark, J. S. Clark, B. Courbaud, C. Fortunel, J. Guillemot, G. Kunstler, G. le Maire, R. Pélissier, N. Rüger, and \underline{G. Vieilledent}.}
\newblock 2023.
\newblock Rethinking the nature of intraspecific variability and its consequences on species coexistence.
\newblock {\em Ecology and Evolution}, 13(3): e9860.
\newblock \textcolor{gray}{\doi{10.1002/ece3.9860}}
  
\setcounter{enumiv}{3}
\bibitem{Vancutsem2021}
\textbf{Vancutsem C., F. Achard, J.-F. Pekel, \underline{G. Vieilledent}, S. Carboni, D. Simonetti, J. Gallego, L. E. O. C. Aragão, and R. Nasi.}
\newblock 2021.
\newblock Long-term (1990--2019) monitoring of forest cover changes in the humid tropics.
\newblock {\em Science Advances}, 7: eabe1603.
\newblock \textcolor{gray}{\doi{10.1126/sciadv.abe1603}}

\setcounter{enumiv}{2}
\bibitem{Strona2018}
\textbf{Strona G., S. D. Stringer, \underline{G. Vieilledent}, Z. Szantoi, J. Garcia-Ulloa, and S. A. Wich.}
\newblock 2018.
\newblock Small room for compromise between oil palm cultivation and primate conservation in Africa.
\newblock {\em Proceedings of the National Academy of Sciences (PNAS)}, 115: 8811--8816.
\newblock \textcolor{gray}{\doi{10.1073/pnas.1804775115}}

\setcounter{enumiv}{1}
\bibitem{Kunstler2016}
\textbf{Kunstler G., D. Falster, D. Coomes, F. Hui, R. Kooyman, D. Laughlin, L. Poorter, M. Vanderwel, \underline{G. Vieilledent}, [...], and M. Westoby.} 
\newblock 2016.
\newblock Plant functional traits have globally consistent effects on competition.
\newblock{\em Nature}, 529: 204--207.
\newblock \textcolor{gray}{\doi{10.1038/nature16476}}

\setcounter{enumiv}{0}
\bibitem{Chave2015}
\textbf{Chave J., M. Réjou-Méchain, A. Búrquez, E. Chidumayo, M. S. Colgan, W. B. C. Delitti, [...], and \underline{G. Vieilledent}.} 
\newblock 2014.
\newblock Improved allometric models to estimate the aboveground biomass of tropical trees. 
\newblock{\em Global Change Biology}, 20: 3177--3190.
\newblock \textcolor{gray}{\doi{10.1111/gcb.12629}}

\end{thebibliography}

\end{document}
