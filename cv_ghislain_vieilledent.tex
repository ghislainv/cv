\documentclass[10pt,a4paper,sans]{moderncv}
\renewcommand*{\bibliographyhead}[1]{}
\moderncvstyle{classic}      
\moderncvcolor{green}             
\usepackage[utf8]{inputenc}
\usepackage[scale=0.8]{geometry}
\usepackage{eurosym}
\usepackage{calc}
\usepackage{lastpage}

\firstname{Ghislain}
\familyname{Vieilledent}
\title{Ecologie -- Statistiques appliquées}              
\address{Piazza Buzzi, 1}{21100 Varese, Italie}
\mobile{+39.329.457.2273}
%\mobile{+33.(0)6.24.62.65.07} 
%\mobile{+261.(0)32.07.235.34}                    
\email{ghislain.vieilledent@cirad.fr}                      
\homepage{ghislain.vieilledent.free.fr}
\extrainfo{né en 1979 à Nantes (36 ans)}
\photo[70pt][0.4pt]{Identite_Australie.png}
%\photo[70pt][0.4pt]{ghislain_5_300x300.png}
\rfoot{\textcolor{gray}{\textit{\small{\thepage/\pageref{LastPage}}}}} 

% to show numerical labels in the bibliography; only useful if you make citations in your resume
\newcounter{enumiv_gv}
\setcounter{enumiv_gv}{0}
\makeatletter
\renewcommand*{\bibliographyitemlabel}{\@biblabel{\arabic{enumiv}}}
\makeatother

\begin{document}
\makecvtitle

\section{Poste actuel}
\cventry{depuis Sept. 2015}{Expert national détaché (END) sur la modélisation de la déforestation en zone tropicale}{European Commission, Joint Research Center} {Ispra (Italie)}{}{}
\cventry{depuis Mars 2009}{Chercheur en écologie et statistiques appliquées}{Cirad, UPR
  Biens et Services des Ecosystèmes Forestiers} {Montpellier (France) et Antananarivo
  (Madagascar)}{}{}

\section{Thèmes de recherche}
\cvline{}{\textbf{Gestion durable des écosystèmes forestiers} via la simulation de la
  dynamique forestière.}
\cvline{}{\textbf{Changements climatiques} (atténuation et adaptation) et \textbf{conservation} des espèces.}
\cvline{}{\textbf{Ecologie des communautés} (approche théorique et expérimentale).}
\cvline{}{\textbf{Nouvelles méthodes statistiques} en écologie.}

\section{Formation et diplômes}

\cvline{2006 -- 2009}{\textbf{Cemagref de Grenoble, AgroParisTech.} \emph{Thèse de doctorat (PhD)} en
  statistiques appliquées à l'écologie forestière. Mécanismes de coexistence du sapin
  pectiné et de l'épicéa commun en forêt de montagne.}

\cvline{2004 -- 2006}{\textbf{Chambre d'Agriculture de Lozère, CLAPE-LR,
  FREDEC-MP.} Ingénieur agronome chargé d'études.}

\cvline{2003 -- 2004}{\textbf{Muséum National d'Histoire Naturelle (MNHN), INA-PG, Paris
  VII.} \emph{Diplôme d'étude approfondi (DEA).} Environnement: Milieux, Techniques et Sociétés
  (EMTS).}

\cvline{2000 -- 2003}{\textbf{Ecole Nationale Supérieure Agronomique de Rennes
  (ENSAR).} \emph{Diplôme d'ingénieur agronome.} Agronomie, écologie quantitative et conservation
  des milieux naturels.}

\cvline{Janvier -- Juin 2001}{\textbf{Escuela Técnica Superior de Ingenieros Agronomos y
  Montes en Córdoba (ETSIAM), Spain.} Semestre d'étude à l'étranger. Sylviculture et
  conservation des habitats méditerranéens.}

\cvline{1996 -- 1999}{\textbf{Lycée Georges Clémenceau, Nantes.} Classes préparatoires
  BCPST (1997 -- 1999) et baccalauréat S option mathématiques (1996 -- 1997) -- mention bien.}

\section{Recherche appliquée / Expertises}

\cvline{2014 -- 2016}{Projet \textbf{BioSceneMada} (coordinateur), Scénarios de la biodiversité à
  Madagascar sous l'effet du changement climatique et de la déforestation,
  \textbf{FRB--FFEM}, 116748 \euro. \url{http://bioscenemada.net}}

\cvline{2012 -- 2013}{Appui à l'Office National de l'Environnement à Madagascar dans la mise
  en place du programme REDD+ pour la gestion et la conservation des forêts, \textbf{AFD},
  47960 \euro.}

\cvline{2012}{Architecture des arbres dans les forêts sèches de l'ouest malgache,
  \textbf{WWF}, 3500 \euro.}

\cvline{2011 -- 2012}{Modélisation de la déforestation (taux et facteurs de déforestation) à
  Madagascar, \textbf{GoodPlanet}, 9000 \euro.}

\cvline{2010 -- 2011}{Développement d'équations allométriques de biomasse pour les forêts
  Malgaches, \textbf{WWF--GoodPlanet}, 8750 \euro.}

\section{Développement de logiciels}

\cvline{\textbf{MCMCpack}}{Contribution au package R \textbf{MCMCpack} à travers le
  développement de fonctions pour l'estimation de modèles mixtes linéaires généralisés
  (glmm): \url{http://cran.r-project.org/package=MCMCpack}}

\cvline{\textbf{hSDM}}{Développement du
  package R \textbf{hSDM} pour les modèles Bayésiens hiérarchiques de distribution des
  espèces: \url{http://hSDM.sf.net}}

\cvline{\textbf{phcfM}}{Développement du package R \textbf{phcfM} pour la modélisation de
  la déforestation d'origine anthropique: \url{http://phcfM.sf.net}}

\cvline{\textbf{twoe}}{Développement du package R et du module Capsis nommé \textbf{twoe}
  pour la simulation de la dynamique des forêts tropicales à partir de données de
  placettes permanentes: \url{http://twoe.sf.net}} 

\section{Compétences en informatique et langues}

\cvline{\textbf{Langages}}{R, C/C++, Python, Java, Bash, SQL}
\cvline{\textbf{SIG}}{GRASS GIS, QGis}
\cvline{\textbf{OS}}{Windows, Linux (Debian)}
\cvline{\textbf{Anglais}}{Très bon niveau lu, écrit et parlé. Séjour de deux mois en
  Caroline du Nord (Duke University).}
\cvline{\textbf{Espagnol}}{Très bon niveau lu, écrit et parlé. Séjours de six mois en
  Espagne et six mois au Pérou.}
\cvline{\textbf{Italien}}{Trois ans d'étude. Bon niveau lu, écrit et parlé. Plusieurs séjours en Italie.}

\section{Enseignements}
\cvline{\textbf{Encadrement}}{Depuis 2009: 1 doctorant, 8 stagiaires de master.}
\cvline{\textbf{Cours}}{Modèles théoriques en écologie. Etudiants en Master 1 Biologie,
  Ecologie, Environnement. Université Joseph Fourier. Grenoble. 3h.}
\cvline{\textbf{TP}}{Modèles théoriques en écologie. Résolution de systèmes d’équations
  différentielles pour l’étude des mécanismes de maintien de la diversité. Etudiants en
  Master 1 Biologie, Ecologie, Environnement. Université Joseph Fourier. Grenoble. 16h.}
\cvline{\textbf{TP}}{Modèles de dynamique forestière. Utilisation du logiciel Capsis et du
  module Samsara. Etudiants en Master II. Ecole d’Ingénieurs FIF-ENGREF. Nancy. 4h.}
\cvline{\textbf{Cours et TP}}{Régressions statistiques et modèles allométriques de
  biomasse. Utilisation du logiciel R. Ingénieurs et techniciens. Office National de
  l'Environnement. Antananarivo. 28h.}

% \section{Centres d'intérêt}
% \cvline{\textbf{Sport}}{Voile, kitesurf, course à pied (trails), escalade}
% \cvline{\textbf{Activités de plein air}}{Randonnée, pêche à la mouche, observations naturalistes}
% \cvline{\textbf{Culture}}{Musique folk (guitare et harmonica) et sudaméricaine (salsa, cumbia), amateur de
% BD}
% \cvline{\textbf{Associatif}}{Promotion des logiciels libres communautaires}

\section{Publications scientifiques}

\vspace{0.3cm}
\subsection{Résumé}

\cvline{}{Depuis 2009, 25 articles scientifiques et 2 chapitres d'ouvrage}
\cvline{}{Liste complète: \url{http://ghislain.vieilledent.free.fr/?page_id=97}}

\vspace{0.3cm}
\subsection{Sélection de dix publications}

\begin{thebibliography}{10}

\setcounter{enumiv}{9}
\bibitem{Vieilledent2016a}
\textbf{Plumptre A.~J., S. Nixon, D. Kujirakwinja, \underline{G. Vieilledent}, R. Critchlow, R. Nishuli, A. Kirkby, E.~A. Williamson, and J.~S Hall.} 
\newblock 2016.
\newblock Catastrophic decline of world’s largest primate: 80\% loss of Grauer’s gorilla (\emph{Gorilla beringei graueri}) population makes it critically endangered.
\newblock{\em PLoS One}, in review.

\setcounter{enumiv}{8}
\bibitem{Vieilledent2016}
\textbf{\underline{Vieilledent G.}, O. Gardi, C. Grinand, C. Burren, M. Andriamanjato, C. Camara, C. J. Gardner, L. Glass, A. Rasolohery, H. Rakoto Ratsimba, V. Gond, and J.-R. Rakotoarijaona.} 
\newblock 2016.
\newblock Bioclimatic envelope models predict a decrease in tropical forest carbon stocks with climate change in Madagascar.
\newblock{\em Journal of Ecology}, in press.
\newblock \textcolor{gray}{[doi: \href{http://dx.doi.org/10.1111/1365-2745.12548}{10.1111/1365-2745.12548}]}

\setcounter{enumiv}{7}
\bibitem{Kunstler2016}
\textbf{Kunstler G., D. Falster, D. Coomes, F. Hui, R. Kooyman, D. Laughlin, L. Poorter, M. Vanderwel, \underline{G. Vieilledent}, S.~J. Wright, M. Aiba, C. Baraloto, J. Caspersen, J. H. C. Cornelissen, S. Gourlet-Fleury, M. Hanewinkel, B. Hérault, J. Kattge, H. Kurokawa, Y. Onoda, J. Penuelas, H. Poorter, M. Uriarte, S. Richardson, P. Ruiz-Benito, I.-F. Sun, G. Ståhl, N. Swenson, J. Thompson, B. Westerlund, C. Wirth, M. Zavala , H. Zeng, J. Zimmerman, N. Zimmermann, and M. Westoby.} 
\newblock 2016.
\newblock Plant functional traits have globally consistent effects on competition.
\newblock{\em Nature}, 529: 204-207.
\newblock \textcolor{gray}{[doi: \href{http://dx.doi.org/10.1038/nature16476}{10.1038/nature16476}]}

\setcounter{enumiv}{6}
\bibitem{Chave2015}
\textbf{Chave J., M. Réjou-Méchain, A. Búrquez, E. Chidumayo, M. S. Colgan, W. B. C. Delitti, A. Duque, T. Eid, P. M. Fearnside, R. C. Goodman, M. Henry, A. Martínez-Yrízar, W. A. Mugasha, H. C. Muller-Landau, M. Mencuccini, B. W. Nelson, A. Ngomanda, E. M. Nogueira, E. Ortiz-Malavassi, R. Pélissier, P. Ploton, C. M. Ryan, J. G. Saldarriaga, and \underline{G. Vieilledent}.} 
\newblock 2014.
\newblock Improved allometric models to estimate the aboveground biomass of tropical trees. 
\newblock{\em Global Change Biology}, 20: 3177-3190.
\newblock \textcolor{gray}{[doi: \href{http://dx.doi.org/10.1111/gcb.12629}{10.1111/gcb.12629}]}

\setcounter{enumiv}{5}
\bibitem{Grinand2013}
\textbf{Grinand C., F. Rakotomalala, V. Gond, R. Vaudry, M. Bernoux, and \underline{G. Vieilledent}.} 
\newblock 2013.
\newblock Estimating deforestation in tropical humid and dry forests in Madagascar from 2000 to 2010 using multi-date Landsat satellite images and the Random Forests classifier.
\newblock{\em Remote Sensing of Environment}, 139:68-80.
\newblock \textcolor{gray}{[doi: \href{http://dx.doi.org/10.1016/j.rse.2013.07.008}{10.1016/j.rse.2013.07.008}]}

\setcounter{enumiv}{4}
\bibitem{Vieilledent2013a}
\textbf{\underline{Vieilledent G.}, C. Cornu, A. Cuní Sanchez, J.-M. Leong Pock-Tsy, and
  P. Danthu.} 
\newblock 2013.
\newblock Vulnerability of baobab species to climate change and effectiveness of the
protected area network in Madagascar: towards new conservation priorities. 
\newblock{\em Biological Conservation}, 166:11-22.
\newblock \textcolor{gray}{[doi: \href{http://dx.doi.org/10.1016/j.biocon.2013.06.007}{10.1016/j.biocon.2013.06.007}]}

\setcounter{enumiv}{3}
\bibitem{Vieilledent2013}
\textbf{\underline{Vieilledent G.}, C. Grinand, and R. Vaudry}
\newblock 2013. 
\newblock Forecasting deforestation and carbon emissions in tropical developing countries
facing demographic expansion: a case study in Madagascar.
\newblock{\em Ecology and Evolution}, 3:1702-1716.
\newblock \textcolor{gray}{[doi: \href{http://dx.doi.org/10.1002/ece3.550}{10.1002/ece3.550}]}

\setcounter{enumiv}{2}
\bibitem{Vieilledent2012}
\textbf{\underline{Vieilledent G.}, R. Vaudry, S.~F.~D. Andriamanohisoa, O.~S. Rakotonarivo, H.~Z.
  Randrianasolo, H.~N. Razafindrabe, C. Bidaud Rakotoarivony, J. Ebeling, and M. Rasamoelina.}
\newblock 2012.
\newblock A universal approach to estimate biomass and carbon stock in tropical
  forests using generic allometric models.
\newblock {\em Ecological Applications}, 22(2):572--583.
\newblock \textcolor{gray}{[doi: \href{http://dx.doi.org/10.1890/11-0039.1}{10.1890/11-0039.1}]}

\setcounter{enumiv}{1}
\bibitem{Vieilledent2010b}
\textbf{\underline{Vieilledent G.}, B. Courbaud, G. Kunstler, J.-F. Dhôte, and J.~S. Clark.}
\newblock 2010.
\newblock Individual variability in tree allometry determines light resource
  allocation in forest ecosystems: a hierarchical Bayesian approach.
\newblock {\em Oecologia}, 163(3):759--773.
\newblock \textcolor{gray}{[doi: \href{http://dx.doi.org/10.1007/s00442-010-1581-9}{10.1007/s00442-010-1581-9}]}

\setcounter{enumiv}{0}
\bibitem{Vieilledent2009a}
\textbf{\underline{Vieilledent G.}, B. Courbaud, G. Kunstler, J.-F. Dhôte, and J.~S. Clark.}
\newblock 2009.
\newblock Biases in the estimation of size-dependent mortality models: advantages of a semiparametric approach.
\newblock {\em Canadian Journal of Forest Research}, 39(8):1430--1443.
\newblock \textcolor{gray}{[doi: \href{http://dx.doi.org/10.1139/X09-047}{10.1139/X09-047}]}

\end{thebibliography}

% \vspace{0.3cm}
% \subsection{2013}

% \begin{thebibliography}{5}

% \setcounter{enumiv}{16}
% \bibitem{Grinand2013}
% \textbf{Grinand C., F. Rakotomalala, V. Gond, R. Vaudry, M. Bernoux and \underline{G. Vieilledent}.} 
% \newblock 2013. 
% \newblock Estimating deforestation in tropical humid and dry forests in Madagascar from
% 2000 to 2010 using multi-date Landsat satellite images and the Random Forests
% classifier. 
% \newblock {\em Remote Sensing of Environment}, 139:68-80.
% \newblock \textcolor{gray}{[doi: \href{http://dx.doi.org/10.1016/j.rse.2013.07.008}{10.1016/j.rse.2013.07.008}]} 

% \setcounter{enumiv}{15}
% \bibitem{Henry2013}
% \textbf{Henry M., A. Bombelli, C. Trotta, A. Alessandrini, L. Birigazzi, G. Sola, \underline{G. Vieilledent},
% P. Santenoise, F. Longuetaud, R. Valentini, N. Picard and L. Saint-André.} 
% \newblock 2013.
% \newblock GlobAllomeTree: International platform for tree allometric equations to support
% volume, biomass and carbon assessment. 
% \newblock {\em iForest – Biogeosciences and Forestry}, 6(1):326.
% \newblock \textcolor{gray}{[doi: \href{http://dx.doi.org/10.3832/ifor0901-006}{10.3832/ifor0901-006}]}

% \setcounter{enumiv}{14}
% \bibitem{Vieilledent2013a}
% \textbf{\underline{Vieilledent G.}, C. Cornu, A. Cuní Sanchez, J.-M. Leong Pock-Tsy and
%   P. Danthu.} 
% \newblock 2013.
% \newblock Vulnerability of baobab species to climate change and effectiveness of the
% protected area network in Madagascar: towards new conservation priorities. 
% \newblock{\em Biological Conservation}, 166:11-22.
% \newblock \textcolor{gray}{[doi: \href{http://dx.doi.org/10.1016/j.biocon.2013.06.007}{10.1016/j.biocon.2013.06.007}]}

% \setcounter{enumiv}{13}
% \bibitem{Vieilledent2013}
% \textbf{\underline{Vieilledent G.}, C. Grinand and R. Vaudry}
% \newblock 2013. 
% \newblock Forecasting deforestation and carbon emissions in tropical developing countries
% facing demographic expansion: a case study in Madagascar.
% \newblock{\em Ecology and Evolution}, 3:1702-1716.
% \newblock \textcolor{gray}{[doi: \href{http://dx.doi.org/10.1002/ece3.550}{10.1002/ece3.550}]}

% \end{thebibliography}

% \vspace{0.3cm}
% \subsection{2012}

% \begin{thebibliography}{6}

% \setcounter{enumiv}{12}
% \bibitem{Razakamanarivo2012}
% \textbf{Razakamanarivo R.~H., A. Razakavololona, M.-A. Razafindrakoto,
%   \underline{G. Vieilledent} and A. Albrecht.}
% \newblock 2012.
% \newblock Below-ground biomass production and allometric relationships of
%   eucalyptus coppice plantation in the central highlands of madagascar.
% \newblock {\em Biomass and Bioenergy}, 45:1-10.
% \newblock \textcolor{gray}{[doi: \href{http://dx.doi.org/10.1016/j.biombioe.2011.01.020}{10.1016/j.biombioe.2011.01.020}]}

% \setcounter{enumiv}{11}
% \bibitem{Kunstler2012}
% \textbf{Kunstler G., S. Lavergne, B. Courbaud, W. Thuiller,
%   \underline{G. Vieilledent}, N.~E. Zimmermann, J. Kattge and D.~A. Coomes.}
% \newblock 2012.
% \newblock Competitive interactions between forest trees are driven by species'
%   trait hierarchy, not phylogenetic or functional similarity: implications for
%   forest community assembly.
% \newblock {\em Ecology Letters}, 15(8):831--840.
% \newblock \textcolor{gray}{[doi: \href{http://dx.doi.org/10.1111/j.1461-0248.2012.01803.x}{10.1111/j.1461-0248.2012.01803.x}]}

% \setcounter{enumiv}{10}
% \bibitem{Vieilledent2012}
% \textbf{\underline{Vieilledent G.}, R. Vaudry, S.~F.~D. Andriamanohisoa, O.~S. Rakotonarivo, H.~Z.
%   Randrianasolo, H.~N. Razafindrabe, C. Bidaud Rakotoarivony, J. Ebeling and
%   M. Rasamoelina.}
% \newblock 2012.
% \newblock A universal approach to estimate biomass and carbon stock in tropical
%   forests using generic allometric models.
% \newblock {\em Ecological Applications}, 22(2):572--583.
% \newblock \textcolor{gray}{[doi: \href{http://dx.doi.org/10.1890/11-0039.1}{10.1890/11-0039.1}]}

% \setcounter{enumiv}{9}
% \bibitem{Asner2012a}
% \textbf{Asner G., J. Mascaro, H. Muller-Landau, \underline{G. Vieilledent},
%   R. Vaudry, M. Rasamoelina, J. Hall and M. van Breugel.}
% \newblock 2012.
% \newblock A universal airborne lidar approach for tropical forest carbon
%   mapping.
% \newblock {\em Oecologia}, 168(4):1147--1160.
% \newblock \textcolor{gray}{[doi: \href{http://dx.doi.org/10.1007/s00442-011-2165-z}{10.1007/s00442-011-2165-z}]}

% \setcounter{enumiv}{8}
% \bibitem{Asner2012}
% \textbf{Asner G., J.~K. Clark, J. Mascaro, R. Vaudry, K.~D. Chadwick,
%   \underline{G. Vieilledent}, M. Rasamoelina, A. Balaji,
%   T.~Kennedy-Bowdoin, L. Maatoug, M.~S. Colgan and D.~E. Knapp.}
% \newblock 2012.
% \newblock Human and environmental controls over aboveground carbon storage in
%   madagascar.
% \newblock {\em Carbon Balance and Management}, 7(1):2.
% \newblock \textcolor{gray}{[doi: \href{http://dx.doi.org/10.1186/1750-0680-7-2}{10.1186/1750-0680-7-2}]}

% \setcounter{enumiv}{7}
% \bibitem{Courbaud2012}
% \textbf{Courbaud B., \underline{G. Vieilledent} and G. Kunstler.}
% \newblock 2012.
% \newblock Intra-specific variability and the competition-colonisation
%   trade-off: coexistence, abundance and stability patterns.
% \newblock {\em Theoretical Ecology}, 5(1):61--71.
% \newblock \textcolor{gray}{[doi: \href{http://dx.doi.org/10.1007/s12080-010-0095-8}{10.1007/s12080-010-0095-8}]}

% \end{thebibliography}

% \vspace{0.3cm}
% \subsection{2011}

% \begin{thebibliography}{4}

% \setcounter{enumiv}{6}
% \bibitem{Fox2011}
% \textbf{Fox J., \underline{G. Vieilledent}, C. Yosi, J. Pokana and R.~J. Keenan.}
% \newblock 2011.
% \newblock Aboveground forest carbon dynamics in papua new guinea: Isolating the
%   influence of selective-harvesting and el Nino.
% \newblock {\em Ecosystems}, 14(8): 1276--1288.
% \newblock \textcolor{gray}{[doi: \href{http://dx.doi.org/10.1007/s10021-011-9480-4}{10.1007/s10021-011-9480-4}]}

% \setcounter{enumiv}{5}
% \bibitem{Ruger2011}
% \textbf{Rüger N., U. Berger, S.~P. Hubbell, \underline{G. Vieilledent} and R.
%   Condit.}
% \newblock 2011.
% \newblock Growth strategies of tropical tree species: Disentangling light and
%   size effects.
% \newblock {\em PLoS ONE}, 6(9):e25330.
% \newblock \textcolor{gray}{[doi: \href{http://dx.doi.org/10.1371/journal.pone.0025330}{10.1371/journal.pone.0025330}]}

% \setcounter{enumiv}{4}
% \bibitem{Albert2011}
% \textbf{Albert C.~H., F. Grassein, F.~M. Schurr, \underline{G. Vieilledent} and
%   C. Violle.}
% \newblock 2011.
% \newblock When and how should intraspecific variability be considered in
%   trait-based plant ecology?
% \newblock {\em Perspectives in Plant Ecology, Evolution and Systematics},
%   13(3):217--225.
% \newblock \textcolor{gray}{[doi: \href{http://dx.doi.org/10.1016/j.ppees.2011.04.003}{10.1016/j.ppees.2011.04.003}]}

% \setcounter{enumiv}{3}
% \bibitem{Kunstler2011}
% \textbf{Kunstler G., C.~H. Albert, B. Courbaud, S. Lavergne,
%   W. Thuiller, \underline{G. Vieilledent}, N.~E. Zimmermann and D.~A.
%   Coomes.}
% \newblock 2011.
% \newblock Effects of competition on tree radial-growth vary in importance but
%   not in intensity along climatic gradients.
% \newblock {\em Journal of Ecology}, 99(1):300--312.
% \newblock \textcolor{gray}{[doi: \href{http://dx.doi.org/10.1111/j.1365-2745.2010.01751.x}{10.1111/j.1365-2745.2010.01751.x}]}

% \end{thebibliography}

% \vspace{0.3cm}
% \subsection{2010}

% \begin{thebibliography}{2}

% \setcounter{enumiv}{2}
% \bibitem{Vieilledent2010b}
% \textbf{\underline{Vieilledent G.}, B. Courbaud, G. Kunstler, J.-F. Dhôte
%   and J.~S. Clark.}
% \newblock 2010.
% \newblock Individual variability in tree allometry determines light resource
%   allocation in forest ecosystems: a hierarchical bayesian approach.
% \newblock {\em Oecologia}, 163(3):759--773.
% \newblock \textcolor{gray}{[doi: \href{http://dx.doi.org/10.1007/s00442-010-1581-9}{10.1007/s00442-010-1581-9}]}

% \setcounter{enumiv}{1}
% \bibitem{Vieilledent2010a}
% \textbf{\underline{Vieilledent G.}, B. Courbaud, G. Kunstler and J.-F. Dhôte.}
% \newblock 2010.
% \newblock {Mortality of Silver Fir and Norway Spruce in the Western Alps – A
%   semi-parametric approach combining size-dependent and growth-dependent
%   mortality}.
% \newblock {\em Annals of Forest Science}, 67(3):305.
% \newblock \textcolor{gray}{[doi: \href{http://dx.doi.org/10.1051/forest/2009112}{10.1051/forest/2009112}]}

% \end{thebibliography}

% \vspace{0.3cm}
% \subsection{2009}

% \begin{thebibliography}{1}

% \setcounter{enumiv}{0}
% \bibitem{Vieilledent2009a}
% \textbf{\underline{Vieilledent G.}, B. Courbaud, G. Kunstler, J.-F. Dhôte
%   and J.~S. Clark.}
% \newblock 2009.
% \newblock Biases in the estimation of size-dependent mortality models:
%   advantages of a semiparametric approach.
% \newblock {\em Canadian Journal of Forest Research}, 39(8):1430--1443.
% \newblock \textcolor{gray}{[doi: \href{http://dx.doi.org/10.1139/X09-047}{10.1139/X09-047}]}

% \end{thebibliography}

% \section{Chapitres d'ouvrage}

% \vspace{0.3cm}
% \subsection{2011}

% \begin{thebibliography}{1}

% \setcounter{enumiv}{0}
% \bibitem{Fox2011a}
% \textbf{Fox J., \underline{Vieilledent G.} and R.~J. Keenan.}
% \newblock 2011.
% \newblock Native forest individual-tree modelling in Papua New Guinea.
% \newblock {\em \emph{in} Native forest individual-tree modelling in Papua New Guinea}
% \newblock by J.C. Fox, R.J. Keenan, C.L. Brack and S. Saulei (eds).
% \newblock ACIAR Proceedings Series, Australian Centre for International Agricultural
% Research (ACIAR), Canberra, 106--120.

% \end{thebibliography}

\end{document}
