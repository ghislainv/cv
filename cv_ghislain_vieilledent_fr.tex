\documentclass[10pt,a4paper,sans]{moderncv}
\renewcommand*{\bibliographyhead}[1]{}
\moderncvstyle{classic}      
\moderncvcolor{green}             
\usepackage[utf8]{inputenc}
\usepackage[english,francais]{babel}
\usepackage[scale=0.8]{geometry}
\usepackage{eurosym}
\usepackage{calc}
\usepackage{lastpage}
\usepackage{amsmath}

\firstname{Ghislain}
\familyname{Vieilledent}
\title{Ecologie -- Statistiques appliquées}
\address{UMR AMAP}{2196, bd. de la Lironde}{34980 Montferriez-sur-Lez (FR)}
\mobile{+33.(0)6.24.62.65.07}                  
\email{ghislain.vieilledent@cirad.fr}                      
\homepage{ecology.ghislainv.fr}
\extrainfo{né en 1979 à Nantes (FR)}
\photo[70pt][0.4pt]{Identite_Australie.png}
%\photo[70pt][0.4pt]{ghislain_5_300x300.png}
\rfoot{\textcolor{gray}{\textit{\small{\thepage/\pageref{LastPage}}}}}

% to show numerical labels in the bibliography; only useful if you make citations in your resume
\newcounter{enumiv_gv}
\setcounter{enumiv_gv}{0}
\makeatletter
\renewcommand*{\bibliographyitemlabel}{\@biblabel{\arabic{enumiv}}}
\makeatother

% doi
\newcommand\doi[1]{[doi: \href{https://doi.org/#1}{#1}]}

\begin{document}
\makecvtitle

\vspace{-0.7cm}
\textit{\small{CV mis à jour le \today.}}

\section{Postes occupés}

\cventry{2019 -- prés.}{Chercheur en écologie}{Cirad, UMR AMAP} {Montpellier (France)}{Etude des forêts tropicales}{}
\cventry{2015 -- 2018}{Expert national détaché (END)}{Centre Commun de Recherche de la Commission Européenne} {Ispra (Italie)}{Modélisation et projection de la déforestation en zone tropicale}{}
\cventry{2009 -- 2012}{Expatriation à Madagascar}{Cirad, dP Forêts et Biodiversité} {Antananarivo (Madagascar)}{Conservation des forêts et de la biodiversité à Madagascar}{}
\cventry{2009 -- 2018}{Chercheur en écologie}{Cirad, UPR Forêts et Sociétés} {Montpellier (France)}{Etude des forêts tropicales}{}

\section{Thèmes de recherche}

\cvline{}{\textbf{Changements climatiques et forêts tropicales} Anticiper l'impact des changement climatiques sur les forêts tropicales et quantifier la part des forêts dans le cycle du carbone à l'échelle globale.}
\cvline{}{\textbf{Ecologie des communautés} Expliquer la distribution des espèces d'arbres et leur coexistence via l'analyse de données d'inventaires en forêts tropicales et l'utilisation de modèles théoriques.}
\cvline{}{\textbf{Gestion durable des écosystèmes forestiers} Améliorer le réalisme des modèles de dynamique forestière en vue de leur utilisation en gestion forestière.}
\cvline{}{\textbf{Nouvelles méthodes statistiques} Développer de nouvelles approches et outils statistiques et informatique pour répondre à de nouvelles questions en écologie.}

\section{Formation et diplômes}

\cvline{2006 -- 2009}{\textbf{Cemagref de Grenoble, AgroParisTech.} \emph{Thèse de doctorat (PhD)} en
  statistiques appliquées à l'écologie forestière. Mécanismes de coexistence du sapin
  pectiné et de l'épicéa commun en forêt de montagne.}

\cvline{2004 -- 2006}{\textbf{Chambre d'Agriculture de la Lozère, CLAPE-LR,
  FREDEC-MP.} Ingénieur agronome chargé d'études.}

\cvline{2003 -- 2004}{\textbf{Muséum National d'Histoire Naturelle (MNHN), INA-PG, Paris
  VII.} \emph{Diplôme d'étude approfondi (DEA).} Environnement: Milieux, Techniques et Sociétés
  (EMTS).}

\cvline{2000 -- 2003}{\textbf{Ecole Nationale Supérieure Agronomique de Rennes
  (ENSAR).} \emph{Diplôme d'ingénieur agronome.} Agronomie, écologie quantitative et conservation
  des milieux naturels.}

\cvline{2001}{\textbf{Escuela Técnica Superior de Ingenieros Agronomos y
  Montes (ETSIAM), Cordoue (Espagne).} Semestre d'étude à l'étranger. Sylviculture et
  conservation des habitats méditerranéens.}

\cvline{1996 -- 1999}{\textbf{Lycée Georges Clémenceau, Nantes.} Classes préparatoires
  BCPST (1997 -- 1999) et baccalauréat S option mathématiques (1996 -- 1997).}

\section{Compétences en informatique et langues}

\cvline{\textbf{Langages}}{R, Python, C/C++, Bash, SQL}

\cvline{\textbf{SIG}}{GRASS GIS, QGis}

\cvline{\textbf{OS}}{Linux (Debian)}

\cvline{\textbf{Anglais}}{Très bon niveau lu, écrit et parlé. Séjour de deux mois en
  Caroline du Nord (Duke University).}

\cvline{\textbf{Espagnol}}{Très bon niveau lu, écrit et parlé. Séjours de six mois en
  Espagne et six mois au Pérou.}\enlargethispage{\baselineskip}

\cvline{\textbf{Italien}}{Bon niveau lu, écrit et parlé. Expatrié trois ans en Italie.}

\newpage

\section{Projets de recherche}

\cvline{}{Depuis 2009: coordination de 3 projets scientifiques, responsable de work-packages pour 1 projet, participation à 4 autres projets et réalisation de 6 travaux d'expertises.}
\cvline{\homepagesymbol}{Descriptif complet: \url{https://ecology.ghislainv.fr/research.html\#projects}.}

\vspace{0.3cm}
\subsection{Coordination de projets}

\cvline{2020 -- 2023}{Projet \textbf{METRADICA}, Utilisation de traits fonctionnels mécanistes pour prédire la vulnérabilité des espèces d'arbres au changement climatique en Guyane française, \textbf{Labex CEBA}, 200 000 \euro. \url{https://www.labex-ceba.fr/projets-strategiques/}}

\cvline{2021 -- 2023}{Projet \textbf{INTRACO}, Rôle de la variabilité intraspécifique dans la coexistence des espèces d'arbres en forêt tropicale, \textbf{FRB-Cesab -- sDiv}, 4 workshops ($\sim$ 40 000 \euro). Page du projet sur le site du \href{https://www.fondationbiodiversite.fr/en/the-frb-in-action/programs-and-projects/le-cesab/intraco/}{\texttt{Cesab}.}}

\cvline{2014 -- 2019}{Projet \textbf{BioSceneMada}, Scénarios d'évolution de la biodiversité à Madagascar sous l'effet du changement climatique et de la déforestation, \textbf{FRB -- FFEM}, 116 748 \euro. \url{https://bioscenemada.cirad.fr}}

\vspace{0.3cm}
\subsection{Responsable de work-packages}

\cvline{2019 -- 2022}{Projet \textbf{RELIQUES}, Effet de la fragmentation des forêts sur la biodiversité en Nouvelle-Calédonie, \textbf{CNRT}, 251 380 \euro. Coordination: Philippe Birnbaum (IRD, UMR AMAP). \url{https://cnrt.nc/reliques}}

\vspace{0.3cm}
\subsection{Participation aux projets}

\cvline{2023 -- 2026}{Projet \textbf{EDENE}, "Ecological Dynamics behind species Extinctions in Novel Ecosystems", \textbf{ANR}, 302 165 \euro. Coordination: Robin Pouteau (IRD, UMR AMAP).}

\cvline{2022 -- 2025}{Projet \textbf{ALT}, "Amazonian Landscapes in Transition", \textbf{ANR}, 659 799 \euro. Coordination: Jérôme Chave (CNRS, Université de Toulouse).}

\cvline{2022 -- 2025}{Projet \textbf{GUARDEN}, "safeGUARDing biodivErsity aNd critical ecosystem services across sectors and scales", \textbf{Union Européenne}, Horizon Europe, 4.5M \euro. Coordination: Pierre Bonnet (Cirad, UMR AMAP).}

\cvline{2019 -- 2023}{Projet \textbf{GAMBAS}, "Generating Advances in Modeling Biodiversity And ecosystem Services (GAMBAS): statistical improvements and ecological relevance of joint species distribution models", \textbf{ANR}, 569 033 \euro. Coordination: Frédéric Mortier (Cirad, UPR Forêts et Sociétés).}

\vspace{0.3cm}
\subsection{Travaux d'expertises}

\cvline{2023}{Validation de l'utilisation de cartes globales de déforestation en Nouvelle-Calédonie sur la période 2000--2010--2020 par photo-interprétation, \textbf{Oeil}, 12 000 \euro.}

\cvline{2022}{Développement du package Python \textbf{riskmapjnr} pour l'obtention de cartes de risque de déforestation selon la méthodologie JNR, \textbf{FAO}, 28 000 \euro.}

\cvline{2012 -- 2013}{Appui technique à l'Office National pour l'Environnement dans la mise en place du programme REDD+ pour la gestion et la conservation des forêts à Madagascar, \textbf{AFD}, 47 960 \euro.}

\cvline{2012}{Architecture des arbres dans les forêts sèches de l'ouest malgache, \textbf{WWF}, 3500 \euro.}

\cvline{2011 -- 2012}{Modélisation de la déforestation (intensité et localisation de la déforestation) à Madagascar, \textbf{GoodPlanet}, 9000 \euro.}

\cvline{2010 -- 2011}{Développement d'équations allométriques de biomasse pour les forêts Malgaches, \textbf{WWF -- GoodPlanet}, 8750 \euro.}

\newpage

\section{Développement de logiciels}

\cvline{}{Depuis 2009: Développement et maintenance de 4 packages R et de 3 packages Python. Contribution à 1 package R. Mise en ligne de 70 répertoires \href{https://github.com/ghislainv}{\texttt{GitHub}} publics et de 6 dépôts \href{https://dataverse.cirad.fr/dataverse/CIRAD/?q=\%28authorName\%3Aghislain+authorName\%3Avieilledent\%29}{\texttt{Dataverse Cirad}}.}
\cvline{\homepagesymbol}{Descriptif complet: \url{https://ecology.ghislainv.fr/software.html}.}

\vspace{0.3cm}
\subsection{Packages R}

\cvline{\textbf{gecevar}}{Le package \textbf{gecevar} permet l'extraction de variables climatiques et environnementales pour une région et une résolution données. \url{https://ecology.ghislainv.fr/gecevar}}

\cvline{\textbf{jSDM}}{Le package \textbf{jSDM} permet d'estimer les paramètres de modèles joints de distribution des espèces. \url{https://ecology.ghislainv.fr/jSDM}}

\cvline{\textbf{hSDM}}{Le package \textbf{hSDM} permet d'estimer les paramètres de modèles Bayésiens hiérarchiques de distribution des espèces. \url{https://ecology.ghislainv.fr/hSDM}}

\cvline{\textbf{twoe}}{\textbf{twoe} (2e) est le nom d'un package R et d'un module Capsis pour la simulation de la dynamique des forêts tropicales à partir de données de parcelles forestières permanentes. \url{https://twoe.sourceforge.net}}

\cvline{\textbf{MCMCpack}}{(contribution) Le package \textbf{MCMCpack} permet l'estimation de modèles mixtes linéaires généralisés (glmm). \url{http://cran.r-project.org/package=MCMCpack}}

%% \cvline{\textbf{phcfM}}{Développement du package R \textbf{phcfM} pour la modélisation de
%%  la déforestation d'origine anthropique. \url{https://sourceforge.net/projects/phcfm/files}}

\vspace{0.3cm}
\subsection{Packages Python}

\cvline{\textbf{forestatrisk}}{Le package \textbf{forestatrisk} permet de modéliser la déforestation et de prédire l'évolution du couvert forestier sous scénarios dans les pays tropicaux. \url{https://ecology.ghislainv.fr/forestatrisk}}

\cvline{\textbf{pywdpa}}{Le package \textbf{pywdpa} permet d'extraire les informations issues de la base de données mondiales des aires protégées (WDPA) pour un pays donné. \url{https://ecology.ghislainv.fr/pywdpa}}

\cvline{\textbf{riskmapjnr}}{Le package \textbf{riskmapjnr} permet d'obtenir des cartes de risque de déforestation en suivant la méthodologie JNR (Jurisdictional Nested REDD+). \url{https://ecology.ghislainv.fr/riskmapjnr}}

\section{Encadrement et enseignement}

\cvline{}{Depuis 2009: 2 post-docs ($+$ 2 en collaboration), 4 doctorants ($+$3 en collaboration), 14 stagiaires de Master 2.}

\cvline{\homepagesymbol}{Descriptif complet: \url{https://ecology.ghislainv.fr/people.html}.}

\cvline{2019}{Utilisation de modèles de distribution des espèces pour prédire leur vulnérabilité au changement climatique. Atelier de renforcement de capacités. Etudiants, techniciens et ingénieurs. Campus Numérique Francophone. Antananarivo. 2j.}

\cvline{2018--2019}{Modélisation et projection de la déforestation en zone tropicale. Atelier de renforcement de capacités. Etudiants, techniciens et ingénieurs. CeRSAE du FOFIFA et Campus Numérique Francophone. Antananarivo. 5j.}

\cvline{2012}{Régressions statistiques et modèles allométriques de
  biomasse. Utilisation du logiciel R. Atelier de renforcement de capacités. Ingénieurs et techniciens. Office National pour
  l'Environnement. Antananarivo. 3.5j.}

\cvline{$<$ 2009}{Modèles théoriques en écologie. Cours théorique. Etudiants en Master 1 Biologie,
  Ecologie, Environnement. Université Joseph Fourier. Grenoble. 3h.}

\cvline{$<$ 2009}{Modèles théoriques en écologie. TP. Résolution de systèmes d'équations
  différentielles pour l’étude des mécanismes de maintien de la diversité. Etudiants en
  Master 1 Biologie, Ecologie, Environnement. Université Joseph Fourier. Grenoble. 16h.}

\cvline{$<$ 2009}{Modèles de dynamique forestière. TP. Utilisation du logiciel Capsis et du
  module Samsara. Etudiants en Master II. Ecole d'Ingénieurs FIF-ENGREF. Nancy. 4h.}

\newpage

\section{Publications scientifiques}

\cvline{}{Depuis 2009: 45 articles publiés dans des revues scientifiques de rang A, 5 pre-prints, 4 chapitres d'ouvrages et 3 articles de vulgarisation scientifique. H-index: 32 (GScholar), 23 (WoS).}
\cvline{\homepagesymbol}{Liste complète: \url{https://ecology.ghislainv.fr/publications.html}.}

\vspace{0.3cm}
\subsection{Sélection de cinq publications en premier auteur}

\begin{thebibliography}{5}

\setcounter{enumiv}{4}
\bibitem{Vieilledent2022}
\textbf{\underline{Vieilledent G.}, C. Vancutsem, C. Bourgoin, P. Ploton, P. Verley, and F. Achard.} 
\newblock 2022.
\newblock Spatial scenario of tropical deforestation and carbon emissions for the 21$^{\text{st}}$ century.
\newblock{\em bioRxiv}, 485306.
\newblock \textcolor{gray}{\doi{10.1101/2022.03.22.485306}}

\setcounter{enumiv}{3}
\bibitem{Vieilledent2018}
\textbf{\underline{Vieilledent G.}, C. Grinand, F. A. Rakotomalala, R. Ranaivosoa, J.-R. Rakotoarijaona, T. F. Allnutt, and F. Achard.} 
\newblock 2018.
\newblock Combining global tree cover loss data with historical national forest-cover maps to look at six decades of deforestation and forest fragmentation in Madagascar.
\newblock{\em Biological Conservation}, 222: 189--197.
\newblock \textcolor{gray}{\doi{10.1016/j.biocon.2018.04.008}}

\setcounter{enumiv}{2}
\bibitem{Vieilledent2016}
\textbf{\underline{Vieilledent G.}, O. Gardi, C. Grinand, C. Burren, M. Andriamanjato, C. Camara, C. J. Gardner, L. Glass, A. Rasolohery, H. Rakoto Ratsimba, V. Gond, and J.-R. Rakotoarijaona.} 
\newblock 2016.
\newblock Bioclimatic envelope models predict a decrease in tropical forest carbon stocks with climate change in Madagascar.
\newblock{\em Journal of Ecology}, 104: 703--715.
\newblock \textcolor{gray}{\doi{10.1111/1365-2745.12548}}

\setcounter{enumiv}{1}
\bibitem{Vieilledent2012}
\textbf{\underline{Vieilledent G.}, R. Vaudry, S.~F.~D. Andriamanohisoa, O.~S. Rakotonarivo, H.~Z.
  Randrianasolo, H.~N. Razafindrabe, C. Bidaud Rakotoarivony, J. Ebeling, and M. Rasamoelina.}
\newblock 2012.
\newblock A universal approach to estimate biomass and carbon stock in tropical
  forests using generic allometric models.
\newblock {\em Ecological Applications}, 22(2):572--583.
\newblock \textcolor{gray}{\doi{10.1890/11-0039.1}}

\setcounter{enumiv}{0}
\bibitem{Vieilledent2010b}
\textbf{\underline{Vieilledent G.}, B. Courbaud, G. Kunstler, J.-F. Dhôte, and J.~S. Clark.}
\newblock 2010.
\newblock Individual variability in tree allometry determines light resource
  allocation in forest ecosystems: a hierarchical Bayesian approach.
\newblock {\em Oecologia}, 163(3):759--773.
\newblock \textcolor{gray}{\doi{10.1007/s00442-010-1581-9}}

\end{thebibliography}

\vspace{0.3cm}
\subsection{Sélection de cinq publications en coauteur}

\begin{thebibliography}{5}

\setcounter{enumiv}{4}
\bibitem{Girard-Tercieux2022}
\textbf{Girard-Tercieux C., I. Maréchaux, A. T. Clark, J. S. Clark, B. Courbaud, C. Fortunel, J. Guillemot, G. Kunstler, G. le Maire, R. Pélissier, N. Rüger, and \underline{G. Vieilledent}.}
\newblock 2022.
\newblock Rethinking the nature of intraspecific variability and its consequences on species coexistence.
\newblock {\em Ecology and Evolution}, in press.
\newblock {\em bioRxiv}, 484259.
\newblock \textcolor{gray}{\doi{10.1101/2022.03.16.484259}}
  
\setcounter{enumiv}{3}
\bibitem{Vancutsem2021}
\textbf{Vancutsem C., F. Achard, J.-F. Pekel, \underline{G. Vieilledent}, S. Carboni, D. Simonetti, J. Gallego, L. E. O. C. Aragão, and R. Nasi.}
\newblock 2021.
\newblock Long-term (1990--2019) monitoring of forest cover changes in the humid tropics.
\newblock {\em Science Advances}, 7: eabe1603.
\newblock \textcolor{gray}{\doi{10.1126/sciadv.abe1603}}

\setcounter{enumiv}{2}
\bibitem{Strona2018}
\textbf{Strona G., S. D. Stringer, \underline{G. Vieilledent}, Z. Szantoi, J. Garcia-Ulloa, and S. A. Wich.}
\newblock 2018.
\newblock Small room for compromise between oil palm cultivation and primate conservation in Africa.
\newblock {\em Proceedings of the National Academy of Sciences (PNAS)}, 115: 8811--8816.
\newblock \textcolor{gray}{\doi{10.1073/pnas.1804775115}}

\setcounter{enumiv}{1}
\bibitem{Kunstler2016}
\textbf{Kunstler G., D. Falster, D. Coomes, F. Hui, R. Kooyman, D. Laughlin, L. Poorter, M. Vanderwel, \underline{G. Vieilledent}, [...], and M. Westoby.} 
\newblock 2016.
\newblock Plant functional traits have globally consistent effects on competition.
\newblock{\em Nature}, 529: 204--207.
\newblock \textcolor{gray}{\doi{10.1038/nature16476}}

\setcounter{enumiv}{0}
\bibitem{Chave2015}
\textbf{Chave J., M. Réjou-Méchain, A. Búrquez, E. Chidumayo, M. S. Colgan, W. B. C. Delitti, [...], and \underline{G. Vieilledent}.} 
\newblock 2014.
\newblock Improved allometric models to estimate the aboveground biomass of tropical trees. 
\newblock{\em Global Change Biology}, 20: 3177--3190.
\newblock \textcolor{gray}{\doi{10.1111/gcb.12629}}

\end{thebibliography}

\end{document}
